\documentclass[base=hide,11pt]{elegantbook}

% Para Linux
\usepackage[utf8]{inputenc}
\usepackage[T1]{fontenc}
\usepackage[spanish]{babel}

\title{Guía de  aprendizaje\\
	Unidad  4.1 \\
	Introducción a la inferencia estadística }
\subtitle{Probabilidad y Estadística}

\author{Daniel Enrique González Gómez}
\institute{Pontificia Universidad Javeriana Cali}
\date{Octubre, 2020}
\version{1.00}
\bioinfo{Area}{Estadística}

% Frase....
% \extrainfo{}

%\logo{logo-blue.png}
\cover{Modulo4.png}
%%%%%%%%%%%%%%%%%%%%%%%%%%%%%%%%%%%%%%%%%%%%%%%%%%%%%%%%%%%%%%%%%%%%%%%%%%%%%%%%%%%%%%%%%%%%%%%
%\usepackage{color}
\usepackage{tcolorbox}
%%\usepackage[margin=0.5in]{geometry}
%\usepackage{amsthm,amssymb,amsfonts}
%%\usepackage{tikz,lipsum,lmodern}
%\usepackage[most]{tcolorbox}
%\usepackage{xcolor}

\definecolor{col1}{rgb}{0.42,0.35,0.80}% magenta 
\definecolor{col2}{rgb}{0.0,0.65,0.31}%   verde
\definecolor{col3}{rgb}{1.0,0.49,0.09}%   naranja
\definecolor{col4}{rgb}{0.0,0.2,0.6}%  azul oscuro 
\definecolor{col5}{rgb}{0.99,0.05,0.21}%  rojo

%---------------------------------------------------------------------------------------------
\newtcolorbox{Box1}[2][]{	   % caja  azul
  	colback=white!95!col1,
	colframe=white!20!col1,	fonttitle=\bfseries,
	colbacktitle=white!10!col1,enhanced,
	attach boxed title to top left={xshift=1cm,	yshift=-2mm},
	title=#2,#1}
%------------------------------------------------------ --------------------------------------
\newtcolorbox{Box2}[2][]{  % caja verde  ok
  	colback=white!95!col2,
colframe=white!20!col2,	fonttitle=\bfseries,
colbacktitle=white!10!col2,enhanced,
attach boxed title to top left={xshift=1cm,	yshift=-2mm},
title=#2,#1}
%-------------------------------------------------------------------------------------------

\newtcolorbox{Box3}[2][]{ % caja naranja
  	colback=white!95!col3,
colframe=white!20!col3,	fonttitle=\bfseries,
colbacktitle=white!10!col3,enhanced,
attach boxed title to top left={xshift=1cm,	yshift=-2mm},
title=#2,#1}

%----------------------------------------------------------------------------------------

\newtcolorbox{Box4}[2][]{  % caja purpura  ok
  	colback=white!95!col4,
colframe=white!20!col4,	fonttitle=\bfseries,
colbacktitle=white!10!col4,enhanced,
attach boxed title to top left={xshift=1cm,	yshift=-2mm},
title=#2,#1}

%----------------------------------------------------------------------------------------
\newtcolorbox{Box5}[2][]{    %
  	colback=white!95!col5,
colframe=white!20!col5,	fonttitle=\bfseries,
colbacktitle=white!10!col5,enhanced,
attach boxed title to top left={xshift=1cm,	yshift=-2mm},
title=#2,#1}
%%%%%%%%%%%%%%%%%%%%%%%%%%%%%%%%%%%%%%%%%%%%%%%%%%%%%%%%%%%%%%%%%%%%%%%%%%%%%%%%%%%%%%%%%
\newtcolorbox{mybox}[2][]{boxsep=1em,left=-0em,
	
	colback=blue!5!white, 
	colframe=blue!75!black, 
	fonttitle=\bfseries\sffamily,
	colbacktitle=blue!85!red!60,enhanced,
	
	attach boxed title to top left={yshift=-3mm,xshift=3mm},
	title=#2,#1}


\newtcolorbox{mybox2}[2][]{%
	colback=bg,
	colframe=blue!75!black,	fonttitle=\bfseries,
	coltitle=blue!75!black,
	colbacktitle=white!5!col5,enhanced,
	attach boxed title to top left={yshift=-1.2mm, xshift=2mm},
	title=#2,#1}
%-----------------------------------------------------------------------------------------
\font\domino=domino
\def\die#1{{\domino#1}}


\setlength{\parindent}{0cm} % no sangrado en los parrafos
\usepackage{hyperref} % insertar links
%%%%%%%%%%%%%%%%%%%%%%%%%%%%%%%%%%%%%%%%%%%%%%%%%%%%%%%%%%%%%%%%%%%%%%%%%%%%%%%%%%%%%%%%%%%%%%%
\begin{document}%%%%%%%%%%%%%%%%%%%%%%%%%%%%%%%%%%%%%%%%%%%%%%%%%%%%%%%%%%%%%%%%%%%%%%%%%%%%%%%%%
%%%%%%%%%%%%%%%%%%%%%%%%%%%%%%%%%%%%%%%%%%%%%%%%%%%%%%%%%%%%%%%%%%%%%%%%%%%%%%%%%%%%%%%%%%%%%%%%%	

\maketitle

\frontmatter
%\tableofcontents
%
\mainmatter
%%%%%%%%%%%%%%%%%%%%%%%%%%%%%%%%%%%%%%%%%%%%%%%%%%%%%%%%%%%%%%%%%%%%%%%%%%%%%%%%%%%%%%%

%\begin{introduction}
%	\item Introducción al tema 
%	\item Objetivos de la unidad
%	\item Fundamentos conceptuales
%	\item Metodología
%	\item Criterios de evaluación
%	\item Fechas de entrega
%\end{introduction}
%%%%%%%%%%%%%%%%%%%%%%%%%%%%%%%%%%%%%%%%%%%%%%%%%%%%%%%%%%%%%%%%%%%%%%%%%%%%%%%%%%%%%%
\section*{1. Introducción}

La inferencia estadística permite generalizar lo hallado en una muestra a toda la población. Para realizar este proceso contamos con dos opciones: Estimación y las Pruebas de hipótesis. El fundamento de estos procesos está relacionado con varios conceptos que se describen y se discuten en esta unidad. En algunos casos serán apoyados con procesos de simulación que permitirán la verificación de cada uno de ellos. \\
 
Dentro de estos conceptos se destacan: población, muestra, censo, muestreo, parámetro, estimador, tamaño de muestra, tipo de muestreo, métodos de estimación de momentos, método de estimación de máxima verosimilitud, propiedades de estimadores como son insesgadez, eficiencia, consistencia, teorema central del limite, media muestral, proporción muestral, varianza muestral, modelos de probabilidad normal, t-Student, ji-cuadrado y F.



%%%%%%%%%%%%%%%%%%%%%%%%%%%%%%%%%%%%%%%%%%%%%%%%%%%%%%%%%%%%%%%%%%%%%%%%%%%%%%%%%%%%%%
\section*{2. Objetivos de la unidad}

% Al finalizar el módulo el estudiante estará en capacidad de RECONOCER, INFERIR y CONSTRASTAR a partir de la estimación através de INTERVALOS DE CONFIANZA  y PRUEBAS DE HIPOTESIS, permitiendo incorporar estas herramientas estadisticas a la toma de decisiones.
Al finalizar esta unidad el estudiante estará en capacidad de RECONOCER los elementos, conceptos y propiedades que intervienen en la inferencia estadística, de tal manera que pueda INFERIR de manera adecuada los resultados obtenidos para la toma de decisiones informadas
%%%%%%%%%%%%%%%%%%%%%%%%%%%%%%%%%%%%%%%%%%%%%%%%%%%%%%%%%%%%%%%%%%%%%%%%%%%%%%%%%%%%%%%%%%%%%%%%%%%%


%%%%%%%%%%%%%%%%%%%%%%%%%%%%%%%%%%%%%%%%%%%%%%%%%%%%%%%%%%%%%%%%%%%%%%%%%%%%%%%%%%%%%%%%%%%%%%%%%%%% 
\section*{3. Duración}
La presente  unidad será desarrollada durante la comprendida entre 19 y el 25 de Octubre.    
Ademas del material suministrado  contaran con el acompañamiento del profesor en tres sesiones (Lunes, Miércoles y Viernes) y de manera asincrónica con  foro de actividades académicas. Los entegables para esta unidad podrán enviarse a través de la plataforma Blackboard hasta el  25 de Octubre.

Para alcanzar los objetivos planteados se propone realizar las siguientes actividades
% 	
%%%%%%%%%%%%%%%%%%%%%%%%%%%%%%%%%%%%%%%%%%%%%%%%%%%%%%%%%%%%%%%%%%%%%%%%%%%%%%%%%%%%%%%%%%%%%%%%%%%
\section*{4. Cronograma de trabajo}


\begin{tabular}{p{4cm}p{10cm}}
	\hline	
	Fecha                   & Actividad	\\
	\hline 	
	{\bf Actividad-1}  \hspace{4cm} Trabajo  individual      & 
	
	En el enlace 
	( 
		% 300MAE005 A
	\href{https://padlet.com/ugp667/wcgi0d93plcp89re}{https://padlet.com/ugp667/wcgi0d93plcp89re})
	
	% 300MAE005 B
	%\href{https://padlet.com/ugp667/qa28jowyeujpb58w}{https://padlet.com/ugp667/qa28jowyeujpb58w}
	
	% 300MAE005 F
	%\href{https://padlet.com/ugp667/ayyd8idz764nmvk7}{{https://padlet.com/ugp667/ayyd8idz764nmvk7}
	
	encontrara un muro donde se recogen los principales conceptos de inferencia estadística. Escoja 5 de ellos y realice su aporte. Despues de escoger el concepto de click en {\bf +} y adicione el texto. Ademas del texto, se debe colocar la referencia bibliografica de donde fue consultado el concepto y tu nombre que realiza el aporte.\\ 
	%Recurso                 &  \\
    %	& Guía 4.1 \\
	Fecha  : & 25 de Octubre de 2020\\
	Hora   : & 23:59 hora local \\
	\hline 
	%\end{tabular}

	%%%%%%%%%%%%%%%%%%%%%%%%%%%%%%%%%%%%%%%%%%%%%%%%%%%%%%%%%%%%%%%%%%%%%
	%\begin{tabular}{p{4cm}p{10cm}}
	%	\hline	
	%	Fecha                   & Actividad	\\
	%	\hline
	{\bf Actividad-2}  \hspace{5cm} Trabajo individual& 
	Verifique el cumplimiento del Teorema Central del Limite para el caso de los modelos que le fueron asignados en las actividades de la unidad 3.3\\
	Recurso            & Laboratorio en R	\\
	Fecha  : & 25 de Octubre de 2020\\
	Hora   : & 23:59 hora local \\
	\hline 
	%%%%%%%%%%%%%%%%%%%%%%%%%%%%%%%%%%%%%%%%%%%%%%%%%%%%%%%%%%%%%%%%%%%%%%
	
	\hline 
\end{tabular}
%%%%%%%%%%%%%%%%%%%%%%%%%%%%%%%%%%%%%%%%%%%%%%%%%%%%%%%%%%%%%%%%%%%%%%%%%%%%%%%%%%%%%%%%%%%%%%%%%%%
\section*{5. Criterios de evaluación}

\begin{itemize}
	\item Reconoce los modelos especiales asociados a contextos  y problemas reales
	\item Utiliza las herramientas estadística apropiadas en el calculo de probabilidades en la solución de problemas en contexto
	\item Utiliza herramientas computacionales que le permita  la solución de problemas en contexto a través de la simulación
\end{itemize}

Los entregables completos y enviados dentro de los tiempos establecidos  otorgarán 15 puntos en cada caso, para un  total de 30 puntos. 



%%%%%%%%%%%%%%%%%%%%%%%%%%%%%%%%%%%%%%%%%%%%%%%%%%%%%%%%%%%%%%%%%%%%%%%%%%%%%%%%%%%%%%%%%%%%%%%%%%%
\section*{6. Entregables}

\begin{itemize}
	\item {\bf Entregable 1}:   \textcolor{col4}{\bf Actividad-1.pdf } 
	\item {\bf Entregable 2}:   \textcolor{col4}{\bf actividad-2.pdf } 
\end{itemize}
\vspace{1cm}

Domingo 25 de Octubre de 2020\\
Hora límite : 23:59  hora  local\\


%%%%%%%%%%%%%%%%%%%%%%%%%%%%%%%%%%%%%%%%%%%%%%%%%%%%%%%%%%%%%%%%%%%%%%%%%%%%%%%%%%%%%%%%%%%%%%%%%%%





\end{document}