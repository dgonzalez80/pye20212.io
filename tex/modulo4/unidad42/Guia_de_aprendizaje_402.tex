\documentclass[base=hide,11pt]{elegantbook}

% Para Linux
\usepackage[utf8]{inputenc}
\usepackage[T1]{fontenc}
\usepackage[spanish]{babel}

\title{Guía de  aprendizaje\\
	Unidad  4.2 Intervalos de confianza}
\subtitle{Probabilidad y Estadística}

\author{Daniel Enrique González Gómez}
\institute{Pontificia Universidad Javeriana Cali}
\date{Octubre, 2020}
\version{1.00}
\bioinfo{Area}{Estadística}

% Frase....
% \extrainfo{}

%\logo{logo-blue.png}
\cover{Modulo4.png}
%%%%%%%%%%%%%%%%%%%%%%%%%%%%%%%%%%%%%%%%%%%%%%%%%%%%%%%%%%%%%%%%%%%%%%%%%%%%%%%%%%%%%%%%%%%%%%%
%\usepackage{color}
\usepackage{tcolorbox}
%%\usepackage[margin=0.5in]{geometry}
%\usepackage{amsthm,amssymb,amsfonts}
%%\usepackage{tikz,lipsum,lmodern}
%\usepackage[most]{tcolorbox}
%\usepackage{xcolor}

\definecolor{col1}{rgb}{0.42,0.35,0.80}% magenta 
\definecolor{col2}{rgb}{0.0,0.65,0.31}%   verde
\definecolor{col3}{rgb}{1.0,0.49,0.09}%   naranja
\definecolor{col4}{rgb}{0.0,0.2,0.6}%  azul oscuro 
\definecolor{col5}{rgb}{0.99,0.05,0.21}%  rojo

%---------------------------------------------------------------------------------------------
\newtcolorbox{Box1}[2][]{	   % caja  azul
  	colback=white!95!col1,
	colframe=white!20!col1,	fonttitle=\bfseries,
	colbacktitle=white!10!col1,enhanced,
	attach boxed title to top left={xshift=1cm,	yshift=-2mm},
	title=#2,#1}
%------------------------------------------------------ --------------------------------------
\newtcolorbox{Box2}[2][]{  % caja verde  ok
  	colback=white!95!col2,
colframe=white!20!col2,	fonttitle=\bfseries,
colbacktitle=white!10!col2,enhanced,
attach boxed title to top left={xshift=1cm,	yshift=-2mm},
title=#2,#1}
%-------------------------------------------------------------------------------------------

\newtcolorbox{Box3}[2][]{ % caja naranja
  	colback=white!95!col3,
colframe=white!20!col3,	fonttitle=\bfseries,
colbacktitle=white!10!col3,enhanced,
attach boxed title to top left={xshift=1cm,	yshift=-2mm},
title=#2,#1}

%----------------------------------------------------------------------------------------

\newtcolorbox{Box4}[2][]{  % caja purpura  ok
  	colback=white!95!col4,
colframe=white!20!col4,	fonttitle=\bfseries,
colbacktitle=white!10!col4,enhanced,
attach boxed title to top left={xshift=1cm,	yshift=-2mm},
title=#2,#1}

%----------------------------------------------------------------------------------------
\newtcolorbox{Box5}[2][]{    %
  	colback=white!95!col5,
colframe=white!20!col5,	fonttitle=\bfseries,
colbacktitle=white!10!col5,enhanced,
attach boxed title to top left={xshift=1cm,	yshift=-2mm},
title=#2,#1}
%%%%%%%%%%%%%%%%%%%%%%%%%%%%%%%%%%%%%%%%%%%%%%%%%%%%%%%%%%%%%%%%%%%%%%%%%%%%%%%%%%%%%%%%%
\newtcolorbox{mybox}[2][]{boxsep=1em,left=-0em,
	
	colback=blue!5!white, 
	colframe=blue!75!black, 
	fonttitle=\bfseries\sffamily,
	colbacktitle=blue!85!red!60,enhanced,
	
	attach boxed title to top left={yshift=-3mm,xshift=3mm},
	title=#2,#1}


\newtcolorbox{mybox2}[2][]{%
	colback=bg,
	colframe=blue!75!black,	fonttitle=\bfseries,
	coltitle=blue!75!black,
	colbacktitle=white!5!col5,enhanced,
	attach boxed title to top left={yshift=-1.2mm, xshift=2mm},
	title=#2,#1}
%-----------------------------------------------------------------------------------------
\font\domino=domino
\def\die#1{{\domino#1}}


\setlength{\parindent}{0cm} % no sangrado en los parrafos
\usepackage{hyperref} % insertar links
%%%%%%%%%%%%%%%%%%%%%%%%%%%%%%%%%%%%%%%%%%%%%%%%%%%%%%%%%%%%%%%%%%%%%%%%%%%%%%%%%%%%%%%%%%%%%%%
\begin{document}%%%%%%%%%%%%%%%%%%%%%%%%%%%%%%%%%%%%%%%%%%%%%%%%%%%%%%%%%%%%%%%%%%%%%%%%%%%%%%%%%
%%%%%%%%%%%%%%%%%%%%%%%%%%%%%%%%%%%%%%%%%%%%%%%%%%%%%%%%%%%%%%%%%%%%%%%%%%%%%%%%%%%%%%%%%%%%%%%%%	

\maketitle

\frontmatter
%\tableofcontents
%
\mainmatter
%%%%%%%%%%%%%%%%%%%%%%%%%%%%%%%%%%%%%%%%%%%%%%%%%%%%%%%%%%%%%%%%%%%%%%%%%%%%%%%%%%%%%%%

%\begin{introduction}
%	\item Introducción al tema 
%	\item Objetivos de la unidad
%	\item Fundamentos conceptuales
%	\item Metodología
%	\item Criterios de evaluación
%	\item Fechas de entrega
%\end{introduction}
%%%%%%%%%%%%%%%%%%%%%%%%%%%%%%%%%%%%%%%%%%%%%%%%%%%%%%%%%%%%%%%%%%%%%%%%%%%%%%%%%%%%%%
\section*{1. Introducción}

La inferencia estadística constituye una de las principales componentes de la metodologia estadistica, encaminados a generar procedimiento para el avance de la ciencia. Con ella se pueden generalizar los
resultados obtenidos en una muestra a toda la población, previo cumplimiento de supuestos establecidos por la estadística (llamada paramétrica) .\\

Antes de empezar es necesario entender los conceptos básicos como: población, censo, parámetro, muestra, muestreo, estimador, estimación, métodos de estimación, propiedades de los estimadores como la insesgadez, la eficiencia, la consistencia .Las distribuciones muestrales : normal, t-student, chi-cuadrado y f de Fisher. Todos trabajados y tratados en la unidad 4.1.\\

El siguiente documento te  oriente sobre los pasos a seguir para alcanzar el objetivo planteado

%%%%%%%%%%%%%%%%%%%%%%%%%%%%%%%%%%%%%%%%%%%%%%%%%%%%%%%%%%%%%%%%%%%%%%%%%%%%%%%%%%%%%%
\section*{2. Objetivos de la unidad}

Al finalizar la unidad el estudiante estará en capacidad de IDENTIFICAR, CALCULAR, INTERPRETAR e INFERIR sobre un intervalo de confianza para estimadores de una y dos poblaciones,
que le permita seleccionar la mejor alternativa para la estimación de un parámetro poblacional, orientados a la toma de decisiones  informadas.
%%%%%%%%%%%%%%%%%%%%%%%%%%%%%%%%%%%%%%%%%%%%%%%%%%%%%%%%%%%%%%%%%%%%%%%%%%%%%%%%%%%%%%%%%%%%%%%%%%%%


%%%%%%%%%%%%%%%%%%%%%%%%%%%%%%%%%%%%%%%%%%%%%%%%%%%%%%%%%%%%%%%%%%%%%%%%%%%%%%%%%%%%%%%%%%%%%%%%%%%% 
\section*{3. Duración}
La presente  unidad será desarrollada desde el 30 de octubre hasta 4 de noviembre de 2020 . Ademas del material suministrado  contaran con el acompañamiento del profesor en tres sesiones (Viernes Lunes y  Miércoles) y de manera asincrónica con  foro de actividades académicas. Los entegables para esta unidad podrán enviarse a través de la plataforma Blackboard hasta el  8 de noviembre de 2020 

Para alcanzar los objetivos planteados se propone realizar las siguientes actividades
% 	
%%%%%%%%%%%%%%%%%%%%%%%%%%%%%%%%%%%%%%%%%%%%%%%%%%%%%%%%%%%%%%%%%%%%%%%%%%%%%%%%%%%%%%%%%%%%%%%%%%%
\section*{4. Cronograma de trabajo}


\begin{tabular}{p{4cm}p{10cm}}
\hline	
Fecha                   & Actividad	\\
\hline 	
{\bf Actividad 1}            & Construir un mapa mental sobre intervalos de confianza elaborado a mano, el cual deberá ser entregado en formato pdf \\
Trabajo individual     & \ \\
Fecha  : & 8 de noviembre de 2020\\
Hora   : & 23:59 hora local \\
\hline 
\end{tabular}
%%%%%%%%%%%%%%%%%%%%%%%%%%%%%%%%%%%%%%%%%%%%%%%%%%%%%%%%%%%%%%%%%%%%
\begin{tabular}{p{4cm}p{10cm}}
	\hline	
	Fecha                   & Actividad	\\
	\hline
{\bf Actividad 2}  & Resuelva las preguntas y problemas planteadas en  el  taller y entregue su solución  en formato  pdf en el enlace correspondiente de  Blackboard\\
Trabajo individual &   \\
Fecha  : & 8 de noviembre de 2020\\
Hora   : & 23:59 hora local \\
\hline 
%%%%%%%%%%%%%%%%%%%%%%%%%%%%%%%%%%%%%%%%%%%%%%%%%%%%%%%%%%%%%%%%%%%%%%

\hline 
\end{tabular}
%%%%%%%%%%%%%%%%%%%%%%%%%%%%%%%%%%%%%%%%%%%%%%%%%%%%%%%%%%%%%%%%%%%%%%%%%%%%%%%%%%%%%%%%%%%%%%%%%%%
%\section*{5. Criterios de evaluación}
%
%\begin{itemize}
%	\item 
%	\item 
%	\item 
%\end{itemize}
%
%Los entregables completos y enviados dentro de los tiempos establecidos  otorgarán 10 puntos en cada caso, para un  total de 30 puntos. 



%%%%%%%%%%%%%%%%%%%%%%%%%%%%%%%%%%%%%%%%%%%%%%%%%%%%%%%%%%%%%%%%%%%%%%%%%%%%%%%%%%%%%%%%%%%%%%%%%%%
%\section*{6. Entregables}
\section*{5. Entregables}
\begin{itemize}
\item {\bf Entregable 1}: Mapa mental sobre intervalos de confianza .
\item {\bf Entregable 2}: Solucionario de problemas sobre intervalos de confianza.  
\end{itemize}
\vspace{1cm}

Domingo 8 de noviembre de 2020\\
Hora límite : 23:59  hora  local


%%%%%%%%%%%%%%%%%%%%%%%%%%%%%%%%%%%%%%%%%%%%%%%%%%%%%%%%%%%%%%%%%%%%%%%%%%%%%%%%%%%%%%%%%%%%%%%%%%%
\end{document}