\documentclass[base=hide,11pt]{elegantbook}

% Para Linux
\usepackage[utf8]{inputenc}
\usepackage[T1]{fontenc}
\usepackage[spanish]{babel}

\title{Guía de  aprendizaje \\
	Unidad  1.2 \\
	Tablas de frecuencia e \\
	Indicadores descriptivas}
\subtitle{Probabilidad y Estadística}

\author{Daniel Enrique González Gómez}
\institute{Pontificia Universidad Javeriana Cali}
\date{Febrero, 2021}
\version{1.00}
\bioinfo{Area}{Estadística}

% Frase....
% \extrainfo{}

%\logo{logo-blue.png}
\cover{banner_101.png}
%%%%%%%%%%%%%%%%%%%%%%%%%%%%%%%%%%%%%%%%%%%%%%%%%%%%%%%%%%%%%%%%%%%%%%%%%%%%%%%%%%%%%%%%%%%%%%%
%\usepackage{color}
\usepackage{tcolorbox}
%%\usepackage[margin=0.5in]{geometry}
%\usepackage{amsthm,amssymb,amsfonts}
%%\usepackage{tikz,lipsum,lmodern}
%\usepackage[most]{tcolorbox}
%\usepackage{xcolor}

\definecolor{col1}{rgb}{0.42,0.35,0.80}% magenta 
\definecolor{col2}{rgb}{0.0,0.65,0.31}%   verde
\definecolor{col3}{rgb}{1.0,0.49,0.09}%   naranja
\definecolor{col4}{rgb}{0.0,0.2,0.6}%  azul oscuro 
\definecolor{col5}{rgb}{0.99,0.05,0.21}%  rojo

%---------------------------------------------------------------------------------------------
\newtcolorbox{Box1}[2][]{	   % caja  azul
  	colback=white!95!col1,
	colframe=white!20!col1,	fonttitle=\bfseries,
	colbacktitle=white!10!col1,enhanced,
	attach boxed title to top left={xshift=1cm,	yshift=-2mm},
	title=#2,#1}
%------------------------------------------------------ --------------------------------------
\newtcolorbox{Box2}[2][]{  % caja verde  ok
  	colback=white!95!col2,
colframe=white!20!col2,	fonttitle=\bfseries,
colbacktitle=white!10!col2,enhanced,
attach boxed title to top left={xshift=1cm,	yshift=-2mm},
title=#2,#1}
%-------------------------------------------------------------------------------------------

\newtcolorbox{Box3}[2][]{ % caja naranja
  	colback=white!95!col3,
colframe=white!20!col3,	fonttitle=\bfseries,
colbacktitle=white!10!col3,enhanced,
attach boxed title to top left={xshift=1cm,	yshift=-2mm},
title=#2,#1}

%----------------------------------------------------------------------------------------

\newtcolorbox{Box4}[2][]{  % caja purpura  ok
  	colback=white!95!col4,
colframe=white!20!col4,	fonttitle=\bfseries,
colbacktitle=white!10!col4,enhanced,
attach boxed title to top left={xshift=1cm,	yshift=-2mm},
title=#2,#1}

%----------------------------------------------------------------------------------------
\newtcolorbox{Box5}[2][]{    %
  	colback=white!95!col5,
colframe=white!20!col5,	fonttitle=\bfseries,
colbacktitle=white!10!col5,enhanced,
attach boxed title to top left={xshift=1cm,	yshift=-2mm},
title=#2,#1}
%%%%%%%%%%%%%%%%%%%%%%%%%%%%%%%%%%%%%%%%%%%%%%%%%%%%%%%%%%%%%%%%%%%%%%%%%%%%%%%%%%%%%%%%%
\newtcolorbox{mybox}[2][]{boxsep=1em,left=-0em,
	
	colback=blue!5!white, 
	colframe=blue!75!black, 
	fonttitle=\bfseries\sffamily,
	colbacktitle=blue!85!red!60,enhanced,
	
	attach boxed title to top left={yshift=-3mm,xshift=3mm},
	title=#2,#1}


\newtcolorbox{mybox2}[2][]{%
	colback=bg,
	colframe=blue!75!black,	fonttitle=\bfseries,
	coltitle=blue!75!black,
	colbacktitle=white!5!col5,enhanced,
	attach boxed title to top left={yshift=-1.2mm, xshift=2mm},
	title=#2,#1}
%-----------------------------------------------------------------------------------------
\font\domino=domino
\def\die#1{{\domino#1}}


\setlength{\parindent}{0cm} % no sangrado en los parrafos
\usepackage{hyperref} % insertar links
%%%%%%%%%%%%%%%%%%%%%%%%%%%%%%%%%%%%%%%%%%%%%%%%%%%%%%%%%%%%%%%%%%%%%%%%%%%%%%%%%%%%%%%%%%%%%%%
\begin{document}%%%%%%%%%%%%%%%%%%%%%%%%%%%%%%%%%%%%%%%%%%%%%%%%%%%%%%%%%%%%%%%%%%%%%%%%%%%%%%%%%
%%%%%%%%%%%%%%%%%%%%%%%%%%%%%%%%%%%%%%%%%%%%%%%%%%%%%%%%%%%%%%%%%%%%%%%%%%%%%%%%%%%%%%%%%%%%%%%%%	

\maketitle

\frontmatter
%\tableofcontents
%
\mainmatter
%%%%%%%%%%%%%%%%%%%%%%%%%%%%%%%%%%%%%%%%%%%%%%%%%%%%%%%%%%%%%%%%%%%%%%%%%%%%%%%%%%%%%%%

%\begin{introduction}
%	\item Introducción al tema 
%	\item Objetivos de la unidad
%	\item Fundamentos conceptuales
%	\item Metodología
%	\item Criterios de evaluación
%	\item Fechas de entrega
%\end{introduction}
%%%%%%%%%%%%%%%%%%%%%%%%%%%%%%%%%%%%%%%%%%%%%%%%%%%%%%%%%%%%%%%%%%%%%%%%%%%%%%%%%%%%%%
\section*{1. Introducción}

El {\bf Análisis Descriptivo} 

permite examinar información contenida en una base de datos, procesarla y mediante la construcción de tablas, indicadores y gráficos, sacar conclusiones, proceso importante en la toma de decisiones y par ampliar el conocimiento sobre un tema de interés.

En este proceso es importante conocer el  tipo de variable ( cualitativa o cuantitativa) y su tipo de escala (nominal, ordinal en el caso de  las cualitativas o de intervalo, de razón en  el caso de las cuantitativas), con  fin de seleccionar la manera adecuada de construir tablas que resuman la información e indicadores en cada caso. \\

En  esta unidad se tratara en primer lugar el resumen de información mediante tablas de frecuencia y en  una segunda parte lo relacionado con los indicadores de posición, centro o tendencia, dispersión e indicadores de forma.\\

%%%%%%%%%%%%%%%%%%%%%%%%%%%%%%%%%%%%%%%%%%%%%%%%%%%%%%%%%%%%%%%%%%%%%%%%%%%%%%%%%%%%%%
\section*{2. Objetivos de la unidad}
%%%%%%%%%%%%%%%%%%%%%%%%%%%%%%%%%%%%%%%%%%%%%%%%%%%%%%%%%%%%%%%%%%%%%%%%%%%%%%%%%%%%
Al finalizar esta unidad el estudiante estará en capacidad de RESUMIR e INTERPRETAR información mediante la construcción de TABLAS DE FRECUENCIA, INDICADORES DESCRIPTIVOS que permitan un correcto análisis de datos. 

%%%%%%%%%%%%%%%%%%%%%%%%%%%%%%%%%%%%%%%%%%%%%%%%%%%%%%%%%%%%%%%%%%%%%%%%%%%%%%%%%%%%%%%%%%%%%%%%%%%% 
\section*{3. Duración}
La presente  unidad será desarrollada durante la  segunda semana del semestre ( 8 al 14 de febrero de 2021). Ademas del material suministrado  contaran con el acompañamiento del profesor en dos sesiones (lunes, miércoles y viernes) y de manera asincrónica con  foro de actividades académicas. Los entregables para esta unidad deberán enviarse a través de la plataforma Blackboard hasta el  14 de febrero. \\

Para alcanzar los objetivos planteados se propone realizar las siguientes actividades
% 	
%%%%%%%%%%%%%%%%%%%%%%%%%%%%%%%%%%%%%%%%%%%%%%%%%%%%%%%%%%%%%%%%%%%%%%%%%%%%%%%%%%%%%%%%%%%%%%%%%%%
\section*{4. Cronograma de trabajo}+ 
\begin{tabular}{p{4cm}p{10cm}}
\hline	
Fecha                   & Actividad	\\
\hline 	
{\bf Actividad 1}       &\\
Trabajo individual      & 
\href{https://dgonzalez80.github.io/index.html}{\bf Taller101.html} ( \href{https://javerianacaliedu-my.sharepoint.com/:u:/g/personal/dgonzalez_javerianacali_edu_co/ESnBpW2t9QxMu-pHe3M4_xoBs_qAIm2bsF6_4fBeqZZW8g?e=hjuTdn}{\textcolor{col3}{\bf Taller101.Rmd}} ) Realice las  actividades consignadas en  el taller 1-01, Adjunte y envíe las respuesta a la plataforma de Blackboard en  formato pdf\\
Recursos                 & \href{https://www.youtube.com/watch?v=dWVs-M7oCh4&t=24s}{ Video indicadores estadísticos}\\  
% video de tablas de frecuencia
&\href{https://www.youtube.com/watch?v=n2UCDZNncls}{Video calculo indicadores con tablas de distribución} \\

&\\
Fecha  : & 14 de febrero 2021\\
Hora   : & 23:59 hora local \\
\hline 
\end{tabular}
%%%%%%%%%%%%%%%%%%%%%%%%%%%%%%%%%%%%%%%%%%%%%%%%%%%%%%%%%%%%%%%%%%%%%
\begin{tabular}{p{4cm}p{10cm}}
%\hline	
%Fecha                   & Actividad	\\
&\\ 
{\bf Actividad 2}  & \\
Trabajo individual &  A partir de la  información contenida en  la base de datos  seleccionada en la {\bf Unidad 1.1}, realice un análisis de al menos una variable cualitativa y una cuantitativa teniendo como soportes las tablas de frecuencia y los indicadores estadísticos correspondiente. \\
Recurso &  Base de datos en formato csv \\
        &  Excel \\
        &  RStudio \\
        & Códigos suministrados en  {\bf Presentación 104}\\
&\\
Fecha  : & 14 de febrero 2021\\
Hora   : & 23:59 hora local \\
\hline
\end{tabular}
%%%%%%%%%%%%%%%%%%%%%%%%%%%%%%%%%%%%%%%%%%%%%%%%%%%%%%%%%%%%%%%%%%%%%%%

%%%%%%%%%%%%%%%%%%%%%%%%%%%%%%%%%%%%%%%%%%%%%%%%%%%%%%%%%%%%%%%%%%%%%%%

 

%%%%%%%%%%%%%%%%%%%%%%%%%%%%%%%%%%%%%%%%%%%%%%%%%%%%%%%%%%%%%%%%%%%%%%%%%%%%%%%%%%%%%%%%%%%%%%%%%%%
\section*{5. Criterios de evaluación}

\begin{itemize}
	\item Reconocer e interpretar las diferentes formas  de resumir  los  datos  a través de tablas de frecuencia para los casos de variables cuantitativas y para variables cualitativas.
	\item Calcular e interpretar los diferentes indicadores de posición, tendencia, dispersión y forma que permitan un correcto análisis de los datos
	\item Utilizar herramientas computacionales para el procesamiento de la información. 
\end{itemize}

Los entregables completos y enviados dentro de los tiempos establecidos  otorgarán 10 puntos en cada caso, para un  total de 20 puntos. 



%%%%%%%%%%%%%%%%%%%%%%%%%%%%%%%%%%%%%%%%%%%%%%%%%%%%%%%%%%%%%%%%%%%%%%%%%%%%%%%%%%%%%%%%%%%%%%%%%%%
\section*{6. Entregables}

\begin{itemize}
\item Entregable 1: \textcolor{col3}{\bf Actividad1.pdf}
\item Entregable 1: \textcolor{col3}{\bf Actividad2.pdf}  	
\end{itemize}
\vspace{1cm}

Domingo 14 de febrero de 2021\\
Hora límite : 23:59  hora  local


%%%%%%%%%%%%%%%%%%%%%%%%%%%%%%%%%%%%%%%%%%%%%%%%%%%%%%%%%%%%%%%%%%%%%%%%%%%%%%%%%%%%%%%%%%%%%%%%%%%
\end{document}