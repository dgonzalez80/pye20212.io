\documentclass[base=hide,11pt]{elegantbook}

% Para Linux
\usepackage[utf8]{inputenc}
\usepackage[T1]{fontenc}
\usepackage[spanish]{babel}

\title{Guía de  aprendizaje No. 1\\
	Unidad  1.3 \\
	Visualización de datos}
\subtitle{Probabilidad y Estadística}

\author{Daniel Enrique González Gómez}
\institute{Pontificia Universidad Javeriana Cali}
\date{Febrero, 2020}
\version{1.00}
\bioinfo{Area}{Estadística}

% Frase....
% \extrainfo{}

%\logo{logo-blue.png}
\cover{banner_101.png}
%%%%%%%%%%%%%%%%%%%%%%%%%%%%%%%%%%%%%%%%%%%%%%%%%%%%%%%%%%%%%%%%%%%%%%%%%%%%%%%%%%%%%%%%%%%%%%%
%\usepackage{color}
\usepackage{tcolorbox}
%%\usepackage[margin=0.5in]{geometry}
%\usepackage{amsthm,amssymb,amsfonts}
%%\usepackage{tikz,lipsum,lmodern}
%\usepackage[most]{tcolorbox}
%\usepackage{xcolor}

\definecolor{col1}{rgb}{0.42,0.35,0.80}% magenta 
\definecolor{col2}{rgb}{0.0,0.65,0.31}%   verde
\definecolor{col3}{rgb}{1.0,0.49,0.09}%   naranja
\definecolor{col4}{rgb}{0.0,0.2,0.6}%  azul oscuro 
\definecolor{col5}{rgb}{0.99,0.05,0.21}%  rojo

%---------------------------------------------------------------------------------------------
\newtcolorbox{Box1}[2][]{	   % caja  azul
  	colback=white!95!col1,
	colframe=white!20!col1,	fonttitle=\bfseries,
	colbacktitle=white!10!col1,enhanced,
	attach boxed title to top left={xshift=1cm,	yshift=-2mm},
	title=#2,#1}
%------------------------------------------------------ --------------------------------------
\newtcolorbox{Box2}[2][]{  % caja verde  ok
  	colback=white!95!col2,
colframe=white!20!col2,	fonttitle=\bfseries,
colbacktitle=white!10!col2,enhanced,
attach boxed title to top left={xshift=1cm,	yshift=-2mm},
title=#2,#1}
%-------------------------------------------------------------------------------------------

\newtcolorbox{Box3}[2][]{ % caja naranja
  	colback=white!95!col3,
colframe=white!20!col3,	fonttitle=\bfseries,
colbacktitle=white!10!col3,enhanced,
attach boxed title to top left={xshift=1cm,	yshift=-2mm},
title=#2,#1}

%----------------------------------------------------------------------------------------

\newtcolorbox{Box4}[2][]{  % caja purpura  ok
  	colback=white!95!col4,
colframe=white!20!col4,	fonttitle=\bfseries,
colbacktitle=white!10!col4,enhanced,
attach boxed title to top left={xshift=1cm,	yshift=-2mm},
title=#2,#1}

%----------------------------------------------------------------------------------------
\newtcolorbox{Box5}[2][]{    %
  	colback=white!95!col5,
colframe=white!20!col5,	fonttitle=\bfseries,
colbacktitle=white!10!col5,enhanced,
attach boxed title to top left={xshift=1cm,	yshift=-2mm},
title=#2,#1}
%%%%%%%%%%%%%%%%%%%%%%%%%%%%%%%%%%%%%%%%%%%%%%%%%%%%%%%%%%%%%%%%%%%%%%%%%%%%%%%%%%%%%%%%%
\newtcolorbox{mybox}[2][]{boxsep=1em,left=-0em,
	
	colback=blue!5!white, 
	colframe=blue!75!black, 
	fonttitle=\bfseries\sffamily,
	colbacktitle=blue!85!red!60,enhanced,
	
	attach boxed title to top left={yshift=-3mm,xshift=3mm},
	title=#2,#1}


\newtcolorbox{mybox2}[2][]{%
	colback=bg,
	colframe=blue!75!black,	fonttitle=\bfseries,
	coltitle=blue!75!black,
	colbacktitle=white!5!col5,enhanced,
	attach boxed title to top left={yshift=-1.2mm, xshift=2mm},
	title=#2,#1}
%-----------------------------------------------------------------------------------------
\font\domino=domino
\def\die#1{{\domino#1}}


\setlength{\parindent}{0cm} % no sangrado en los parrafos
\usepackage{hyperref} % insertar links
%%%%%%%%%%%%%%%%%%%%%%%%%%%%%%%%%%%%%%%%%%%%%%%%%%%%%%%%%%%%%%%%%%%%%%%%%%%%%%%%%%%%%%%%%%%%%%%
\begin{document}%%%%%%%%%%%%%%%%%%%%%%%%%%%%%%%%%%%%%%%%%%%%%%%%%%%%%%%%%%%%%%%%%%%%%%%%%%%%%%%%%
%%%%%%%%%%%%%%%%%%%%%%%%%%%%%%%%%%%%%%%%%%%%%%%%%%%%%%%%%%%%%%%%%%%%%%%%%%%%%%%%%%%%%%%%%%%%%%%%%	

\maketitle

\frontmatter
%\tableofcontents
%
\mainmatter
%%%%%%%%%%%%%%%%%%%%%%%%%%%%%%%%%%%%%%%%%%%%%%%%%%%%%%%%%%%%%%%%%%%%%%%%%%%%%%%%%%%%%%%%
%%%%%%%%%%%%%%%%%%%%%%%%%%%%%%%%%%%%%%%%%%%%%%%%%%%%%%%%%%%%%%%%%%%%%%%%%%%%%%%%%%%%%%
\section*{1. Introducción}

La representación grafica de información constituye una de las herramientas mas importantes de la estadística. Con ella podemos observar lo  ocurrido  en  el pasado, el  presente y lo que  podría ocurrir en el futuro y de esta forma orientar nuestras decisiones. \\

Es necesario realizar una  correcta visualización de los datos y para ello requerimos seleccionar la grafica o representación apropiada, conociendo el tipo de variable, su escala de medición y sobre todo lo que queremos resaltar en ella. \\

Las graficas pueden formar parte de un informe en un estudio, para lo cual es necesario conocer los lineamientos exigidos por las normas que los rigen AP, EEE, entre otros. En todos los casos los gráficos deben tener una  enumeración que los permita citar, titulo que permita visualizar lo que están  representando, fuente,  que indique de donde  son extraídos los datos que  la originan. En esta unidad se trataran las diferentes  formas de visualización de datos.

%%%%%%%%%%%%%%%%%%%%%%%%%%%%%%%%%%%%%%%%%%%%%%%%%%%%%%%%%%%%%%%%%%%%%%%%%%%%%%%%%%%%%
\section*{2. Objetivos de la unidad}

Al finalizar la unidad los estudiantes estarán  en  capacidad de  RECONOCER los  tipos de gráficos estadísticos y podrán REPRESENTAR e INTERPRETAR  información a  través  de las diferentes formas, que les permita complementar un  análisis de datos estadístico.

%%%%%%%%%%%%%%%%%%%%%%%%%%%%%%%%%%%%%%%%%%%%%%%%%%%%%%%%%%%%%%%%%%%%%%%%%%%%%%%%%%%%%%%%%%%%%%%%%%%%%
%
%
%%%%%%%%%%%%%%%%%%%%%%%%%%%%%%%%%%%%%%%%%%%%%%%%%%%%%%%%%%%%%%%%%%%%%%%%%%%%%%%%%%%%%%%%%%%%%%%%%%%%% 
\section*{3. Duración}
La presente  unidad será desarrollada durante la  tercera semana del semestre ( 15 al 21 de febrero de 2021 ). Ademas del material suministrado  contaran con el acompañamiento del profesor en tres sesiones (Lunes, Miércoles y Viernes) y de manera asincrónica con  foro de actividades académicas. Los entregables para esta unidad podrán enviarse a través de la plataforma Blackboard hasta el  21 de febrero.

Para alcanzar los objetivos planteados se propone realizar las siguientes actividades
% 	
%%%%%%%%%%%%%%%%%%%%%%%%%%%%%%%%%%%%%%%%%%%%%%%%%%%%%%%%%%%%%%%%%%%%%%%%%%%%%%%%%%%%%%%%%%%%%%%%%%%%
\section*{4. Cronograma de trabajo}


\begin{tabular}{p{4cm}p{10cm}}
\hline	
Fecha                   & Actividad	\\
\hline 	
{\bf Actividad 1}       & \\
Trabajo individual      & A partir de la base de datos trabajada en las dos unidades anteriores construya un {\bf tablero (Dashboard) } que contenga gráficos e indicadores de las variables estadísticas contenidas  en la base de datos seleccionada en la Actividad 1 de la unidad 1.1. Realice una descripción con los resultados obtenidos en el tablero. \\
 
Recurso                 &  Código U1-3.R  y Tablero-M1 \\
                        & \href{https://rstudio.com/wp-content/uploads/2016/10/r-cheat-sheet-3.pdf}{Resumen R base}\\
                        &  \href{https://rmarkdown.rstudio.com/flexdashboard/index.html}{Dashboard en R} \\
                        &  \href{https://rstudio.com/wp-content/uploads/2015/03/ggplot2-cheatsheet.pdf}{Resumen ggplot2}\\
                        & \href{https://rstudio.com/wp-content/uploads/2015/02/rmarkdown-cheatsheet.pdf}{Resumen RMarkdown} \\
                        & \\
Fecha  :                & 21 de febrero de 2021\\
Hora   :                & 23:59 hora local \\
\hline 
\end{tabular}
%%%%%%%%%%%%%%%%%%%%%%%%%%%%%%%%%%%%%%%%%%%%%%%%%%%%%%%%%%%%%%%%%%%%%%
\begin{tabular}{p{4cm}p{10cm}}
%	\hline	
%	Fecha                   & \textcolor{col3}{\bf Actividad 2-U1.3:} \\
%	\hline 	
	{\bf Actividad 2}  & \\
	Trabajo individual &  Construya un {\bf mapa mental} del Módulo 1  y adjunto el link del trabajo (en caso de ser construido en una herramienta en linea) o un pdf  que resuma los conceptos más importantes del Modulo 1. \\
	Recurso            &  \href{https://javerianacaliedu-my.sharepoint.com/:b:/g/personal/dgonzalez_javerianacali_edu_co/ESNCJjqQY8dGlgqu2hm7GTwBq6Fi7GGwB88ul_vw1BWMgg?e=AwTRZM}{Capitulo 1} \\
	& Presentaciones y videos clases \\
	& \\
	& \\
	Fecha  :                & 21 de febrero de 2021\\
	Hora   :                & 23:59 hora local \\
	\hline 
\end{tabular}
%%%%%%%%%%%%%%%%%%%%%%%%%%%%%%%%%%%%%%%%%%%%%%%%%%%%%%%%%%%%%%%%%%%%%%%
\section*{5. Criterios de evaluación}


\begin{itemize}
	\item Reconocer los principales elementos de un  análisis descriptivo de datos y su incidencia en la toma de decisiones.
	\item Reconocer e identificar los diferentes tipos de representaciones graficas y sus respectivas interpretación.
	\item Utilizar herramientas computacionales para el procesamiento de la información. 
\end{itemize}

%%%%%%%%%%%%%%%%%%%%%%%%%%%%%%%%%%%%%%%%%%%%%%%%%%%%%%%%%%%%%%%%%%%%%%%%%%%%%%%%%%%%%%%%%%%%%%%%%%%%
\section*{6. Entregables}

\begin{itemize}
\item {\bf Entregable 1}:\hspace{0.5cm} \textcolor{col3}{\bf Tablero.pdf} o {\bf Tablero} Acompañado de descripción de los resultados 
\item {\bf Entregable 2}: \hspace{0.5cm}\textcolor{col3}{\bf Actividad1-u13.pdf} \hspace{0.5cm} Mapa mental 
\end{itemize}  

\vspace{.3cm}

Domingo 21 de febrero de 2021\\
Hora límite : 23:59  hora  local
%

%%%%%%%%%%%%%%%%%%%%%%%%%%%%%%%%%%%%%%%%%%%%%%%%%%%%%%%%%%%%%%%%%%%%%%%%%%%%%%%%%%%%%%%%%%%%%%%%%%%
\end{document}

Los valores atípicos produce este efecto