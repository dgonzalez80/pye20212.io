\documentclass[base=hide,11pt]{elegantbook}

% Para Linux
\usepackage[utf8]{inputenc}
\usepackage[T1]{fontenc}
\usepackage[spanish]{babel}

\title{Guía de  aprendizaje\\
	Unidad  3.2 Variables aleatorias conjuntas}
\subtitle{Probabilidad y Estadística}

\author{Daniel Enrique González Gómez}
\institute{Pontificia Universidad Javeriana Cali}
\date{Marzo, 2021}
\version{1.00}
\bioinfo{Area}{Estadística}

% Frase....
% \extrainfo{}

%\logo{logo-blue.png}
\cover{Modulo3Z.png}
%%%%%%%%%%%%%%%%%%%%%%%%%%%%%%%%%%%%%%%%%%%%%%%%%%%%%%%%%%%%%%%%%%%%%%%%%%%%%%%%%%%%%%%%%%%%%%%
%\usepackage{color}
\usepackage{tcolorbox}
%%\usepackage[margin=0.5in]{geometry}
%\usepackage{amsthm,amssymb,amsfonts}
%%\usepackage{tikz,lipsum,lmodern}
%\usepackage[most]{tcolorbox}
%\usepackage{xcolor}

\definecolor{col1}{rgb}{0.42,0.35,0.80}% magenta 
\definecolor{col2}{rgb}{0.0,0.65,0.31}%   verde
\definecolor{col3}{rgb}{1.0,0.49,0.09}%   naranja
\definecolor{col4}{rgb}{0.0,0.2,0.6}%  azul oscuro 
\definecolor{col5}{rgb}{0.99,0.05,0.21}%  rojo

%---------------------------------------------------------------------------------------------
\newtcolorbox{Box1}[2][]{	   % caja  azul
  	colback=white!95!col1,
	colframe=white!20!col1,	fonttitle=\bfseries,
	colbacktitle=white!10!col1,enhanced,
	attach boxed title to top left={xshift=1cm,	yshift=-2mm},
	title=#2,#1}
%------------------------------------------------------ --------------------------------------
\newtcolorbox{Box2}[2][]{  % caja verde  ok
  	colback=white!95!col2,
colframe=white!20!col2,	fonttitle=\bfseries,
colbacktitle=white!10!col2,enhanced,
attach boxed title to top left={xshift=1cm,	yshift=-2mm},
title=#2,#1}
%-------------------------------------------------------------------------------------------

\newtcolorbox{Box3}[2][]{ % caja naranja
  	colback=white!95!col3,
colframe=white!20!col3,	fonttitle=\bfseries,
colbacktitle=white!10!col3,enhanced,
attach boxed title to top left={xshift=1cm,	yshift=-2mm},
title=#2,#1}

%----------------------------------------------------------------------------------------

\newtcolorbox{Box4}[2][]{  % caja purpura  ok
  	colback=white!95!col4,
colframe=white!20!col4,	fonttitle=\bfseries,
colbacktitle=white!10!col4,enhanced,
attach boxed title to top left={xshift=1cm,	yshift=-2mm},
title=#2,#1}

%----------------------------------------------------------------------------------------
\newtcolorbox{Box5}[2][]{    %
  	colback=white!95!col5,
colframe=white!20!col5,	fonttitle=\bfseries,
colbacktitle=white!10!col5,enhanced,
attach boxed title to top left={xshift=1cm,	yshift=-2mm},
title=#2,#1}
%%%%%%%%%%%%%%%%%%%%%%%%%%%%%%%%%%%%%%%%%%%%%%%%%%%%%%%%%%%%%%%%%%%%%%%%%%%%%%%%%%%%%%%%%
\newtcolorbox{mybox}[2][]{boxsep=1em,left=-0em,
	
	colback=blue!5!white, 
	colframe=blue!75!black, 
	fonttitle=\bfseries\sffamily,
	colbacktitle=blue!85!red!60,enhanced,
	
	attach boxed title to top left={yshift=-3mm,xshift=3mm},
	title=#2,#1}


\newtcolorbox{mybox2}[2][]{%
	colback=bg,
	colframe=blue!75!black,	fonttitle=\bfseries,
	coltitle=blue!75!black,
	colbacktitle=white!5!col5,enhanced,
	attach boxed title to top left={yshift=-1.2mm, xshift=2mm},
	title=#2,#1}
%-----------------------------------------------------------------------------------------
\font\domino=domino
\def\die#1{{\domino#1}}


\setlength{\parindent}{0cm} % no sangrado en los parrafos
\usepackage{hyperref} % insertar links
%%%%%%%%%%%%%%%%%%%%%%%%%%%%%%%%%%%%%%%%%%%%%%%%%%%%%%%%%%%%%%%%%%%%%%%%%%%%%%%%%%%%%%%%%%%%%%%
\begin{document}%%%%%%%%%%%%%%%%%%%%%%%%%%%%%%%%%%%%%%%%%%%%%%%%%%%%%%%%%%%%%%%%%%%%%%%%%%%%%%%%%
%%%%%%%%%%%%%%%%%%%%%%%%%%%%%%%%%%%%%%%%%%%%%%%%%%%%%%%%%%%%%%%%%%%%%%%%%%%%%%%%%%%%%%%%%%%%%%%%%	

\maketitle

\frontmatter
%\tableofcontents
%
\mainmatter
%%%%%%%%%%%%%%%%%%%%%%%%%%%%%%%%%%%%%%%%%%%%%%%%%%%%%%%%%%%%%%%%%%%%%%%%%%%%%%%%%%%%%%
%%%%%%%%%%%%%%%%%%%%%%%%%%%%%%%%%%%%%%%%%%%%%%%%%%%%%%%%%%%%%%%%%%%%%%%%%%%%%%%%%%%%%%
\section*{1. Introducción}
El concepto de variable  aleatoria  conjunta y sus principales características como la correlación constituyen las bases para aplicaciones estadísticas como los modelos de regresión.  \\ 

En esta  unidad se presentan las principales características de las variables aleatorias conjuntas, mas precisamente bivariadas $(X,Y)$\\



%%%%%%%%%%%%%%%%%%%%%%%%%%%%%%%%%%%%%%%%%%%%%%%%%%%%%%%%%%%%%%%%%%%%%%%%%%%%%%%%%%%%%%
\section*{2. Objetivos de la unidad}

Al finalizar la unidad los estudiantes estarán  en  capacidad de IDENTIFICAR, CALCULAR y APLICAR los conceptos y las principales características  de las VARIABLE ALEATORIA CONJUNTAS para representar situaciones reales en la búsqueda de soluciones a problemas o valorar los riesgos que se pueden presentar en la toma de decisiones.
%%%%%%%%%%%%%%%%%%%%%%%%%%%%%%%%%%%%%%%%%%%%%%%%%%%%%%%%%%%%%%%%%%%%%%%%%%%%%%%%%%%%%%%%%%%%%%%%%%%%


%%%%%%%%%%%%%%%%%%%%%%%%%%%%%%%%%%%%%%%%%%%%%%%%%%%%%%%%%%%%%%%%%%%%%%%%%%%%%%%%%%%%%%%%%%%%%%%%%%%% 
\section*{3. Duración}
La presente  unidad será desarrollada durante la semana séptima del semestre ( 24 de marzo al 7 de abril de 2021). Ademas del material suministrado  contaran con el acompañamiento del profesor en tres sesiones (Lunes, Miércoles y Viernes) y de manera asincrónica con  foro de actividades académicas. Los entregables para esta unidad podrán enviarse a través de la plataforma Blackboard hasta el  07 de abril.

Para alcanzar los objetivos planteados se propone realizar las siguientes actividades
% 	
%%%%%%%%%%%%%%%%%%%%%%%%%%%%%%%%%%%%%%%%%%%%%%%%%%%%%%%%%%%%%%%%%%%%%%%%%%%%%%%%%%%%%%%%%%%%%%%%%%%
\section*{4. Cronograma de trabajo}


\begin{tabular}{p{4cm}p{10cm}}
\hline	
Fecha                   & {\bf Actividad}	\\
\hline 	
\textcolor{col4}{\bf Actividad-1} \hspace{2cm} Trabajo en grupo.  &
\textcolor{col4}{\bf Taller 3.2:} El siguiente taller recoge los principales conceptos relacionados con variables aleatorias, tanto discretas como continuas y sus aplicaciones en la estimación de probabilidades. Actividad realizada en grupos de dos estudiantes. \\
{\bf Recursos}: & {\bf Guía 3.2} \\
          & {\bf Capitulo  2 Navidi}.  (Navidi, W. estadística para ingenieros y científicos. Editorial Mc Graw Hill. 2006. )\\
          & {\bf Capítulos  3 y 4 Walpole}. (Walpole, R. Myers, R. Myers, S. Ye, K. Probabilidad y Estadística para ingeniería y Ciencias.8 Ed. Editorial Pearson. 2007)\\
Fecha  : & 07 de abril  de 2021\\
Hora   : & 23:59 hora local \\
\hline 
%\end{tabular}


.

%%%%%%%%%%%%%%%%%%%%%%%%%%%%%%%%%%%%%%%%%%%%%%%%%%%%%%%%%%%%%%%%%%%%%
%\begin{tabular}{p{4cm}p{10cm}}
%	\hline	
%	Fecha                   & Actividad	\\
%	\hline

%%%%%%%%%%%%%%%%%%%%%%%%%%%%%%%%%%%%%%%%%%%%%%%%%%%%%%%%%%%%%%%%%%%%%%
\end{tabular}
%%%%%%%%%%%%%%%%%%%%%%%%%%%%%%%%%%%%%%%%%%%%%%%%%%%%%%%%%%%%%%%%%%%%%%%%%%%%%%%%%%%%%%%%%%%%%%%%%%%
\section*{5. Criterios de evaluación}

\begin{itemize}
	\item Reconoce las principales características de las variables conjuntas 
	\item Utiliza las herramientas estadística apropiadas en el calculo de probabilidades en la solución de problemas en contexto
	\item Utiliza herramientas computacionales que le permiten representar y calcular probabilidades en la solución de problemas en contexto
	\item Interpreta y expresa el desarrollo de los problemas con el lenguaje propio de la estadística
\end{itemize}

%%%%%%%%%%%%%%%%%%%%%%%%%%%%%%%%%%%%%%%%%%%%%%%%%%%%%%%%%%%%%%%%%%%%%%%%%%%%%%%%%%%%%%%%%%%%%%%%%%%
\section*{6. Entregables}

\begin{itemize}
\item {\bf Entregable}: \textcolor{col4}{\bf Taller 302.pdf :} El taller tiene un valor de 30 puntos si es resuelto completamente y enviado dentro de los tiempos establecidos
\end{itemize}
\vspace{1cm}

\textcolor{col4}{\bf Miercoles 07 de abril de 2021}\\
\textcolor{col4}{\bf Hora límite : 23:59  hora  local}



%%%%%%%%%%%%%%%%%%%%%%%%%%%%%%%%%%%%%%%%%%%%%%%%%%%%%%%%%%%%%%%%%%%%%%%%%%%%%%%%%%%%%%%%%%%%%%%%%%%
\end{document}